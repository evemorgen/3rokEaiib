\newpage
\section{PSPEC}
\begin{multicols}{2}
\subsection{Spis procesów}

\begin{enumerate}
	\item Wydanie karnetu
    	\begin{enumerate}
        	\item Wydanie Karnetu
            	\begin{enumerate}
                	\item Wybór rodzaju karnetu
                    \item Generowanie id karty
                    \item Przypisanie id do karty
                    \item Obliczanie końcowej ceny karnetu
                \end{enumerate}
            \item Obliczenie zniżki
            	\begin{enumerate}
                	\item Zniżki wiekowej
                    \item Zniżki grupowej
                \end{enumerate}
        \end{enumerate}
    \item Dział obsługi klienta
    	\begin{enumerate}
        	\item Zwrot karnetu
            	\begin{enumerate}
                	\item Sprawdzenie ilości punktów
                    \item Weryfikacja aktywacji karnetu
                \end{enumerate}
            \item Drukowanie faktur
        \end{enumerate}
    \item Obsługa szkółki narciarskiej
    	\begin{enumerate}
        	\item Rezerwacja instruktora
            \item Obliczanie ceny zajęć
            \item Prowadzenie rejestru zajęć
        \end{enumerate}
    \item Obsługa wypożyczalni
    	\begin{enumerate}
        	\item Wynajęcie sprzętu
            	\begin{enumerate}
                	\item Stworzenie kodu wypożyczenia
                    \item Zapisanie wypożyczenia
                \end{enumerate}
            \item Obliczenie ceny wypożyczenia
            \item Przyjęcie zwracanego sprzętu
    	\end{enumerate}
    \item Obsługa serwisu
    	\begin{enumerate}
        	\item Serwisowanie sprzętu
            	\begin{enumerate}
                	\item Przyjęcie zlecenia
                    \item Stworzenie kodu zlecenia
                \end{enumerate}
            \item Obliczanie ceny wypożyczenia
            \item Wydanie sprzętu po serwisowaniu
        \end{enumerate}
    \item Obsługa bramek
    	\begin{enumerate}
        	\item Identyfikacja karnetu
            \item Zapisanie czasu odbicia
            \item Odjęcie punktów
            \item Podjęcie decyzji o przepuszczeniu
        \end{enumerate}
    \item Obsługa płatności
    	\begin{enumerate}
        	\item Wybór sposobu płatności
            \item Zapłata gotówką
            	\begin{enumerate}
                	\item Wpłata gotówki
                \end{enumerate}
            \item Zapłata kartą
            	\begin{enumerate}
                	\item Zrealizowanie płatności kartą
                \end{enumerate}
            \item Zapłata online
            	\begin{enumerate}
                	\item Zrealizowanie płatności online
                \end{enumerate}
            \item Odnotowanie zapłaty
        \end{enumerate}
\end{enumerate}
\end{multicols}

\newpage
\subsection{Karnety}
\subsubsection{Wybór rodzaju karnetu.}
\begin{algorithm}[caption={1.1.1 Wybór rodzaju karnetu.}, label={alg1}]
 input: zadanieKupnaKarnetu(rodzajKarnetu, czas=0)
 output: wybranyKarnet, kwotaDoZaplaty
 
begin
  if rodzajKarnetu == "czasowy" 
    wybranyKarnet $gets$ "CZAS"
  else if rodzajKarnetu == "na punkty" 
    wybranyKarnet $gets$ "PKT"
  end if

  if ulga == "normalny" 
    wybranyKarnet $gets$ "N_" + wybranyKarnet
  else if ulga == "wiek" 
    wybranyKarnet $gets$ "W_" + wybranyKarnet
  else if ulga == "grupa" 
    wybranyKarnet $gets$ "G_" + wybranyKarnet
  end if

  id $gets$ generowanieIdKarty(wybranyKarnet)
  cena $gets$ obliczenieCenyKarnetu(wybranyKarnet)
  karnet $gets$ przypiszId(id)
  return karnet, cena
end
\end{algorithm}

\subsubsection{Generowanie id karty}
\begin{algorithm}[caption={1.1.2 Generowanie id karty}, label={alg1}]
  input: wybranyKarnetdaneOsobowe
  output: id

begin
  idTemp $gets$ md5(danePersonalne + czas.now)
  id $gets$ idTemp + wybranyKarnet
  przypiszId()
end
\end{algorithm}

\subsubsection{Przypisanie id do karnetu}
\begin{algorithm}[caption={1.1.3 Przypisanie id do karnetu}, label={alg1}]
  input: id
  output: karnet
  
begin
  karnet $gets$ FALSE
  dodajKarnet(id)
  if dodajKarnet(id) == true 
    karnet $gets$ true
  else
    przypiszId()
end
\end{algorithm}

\subsubsection{Obliczenie ceny karnetu}
\begin{algorithm}[caption={1.1.4 Obliczenie ceny karnetu}, label={alg1}]
input: wybranyKarnet
  output: kwotaDoZaplaty
  
begin
  if ulga == "normalny" 
    kwotaDoZaplaty $gets$ kwotaDoZaplaty
  else if ulga == "ulgowy" 
    kwotaDoZaplaty $gets$ 0.7 * kwotaDoZaplaty
else if ulga == "grupowy" 
    kwotaDoZaplaty $gets$ 0.9 * kwotaDoZaplaty
  end if
  wyborSposobuPlatnosci()
end
\end{algorithm}

\subsubsection{Zwrot karnetu. Sprawdzenie ilości punktów}
\begin{algorithm}[caption={2.1.1 Zwrot karnetu.Sprawdzenie ilości punktów}, label={alg1}]
  input: karnet
  output: kwotaDoZwrotu
  
begin
  if karnet.iloscPunktow > 0 
    kwotaDoZwrotu $gets$ 10 + karnet.iloscPunktow * 2
  else 
    kwotaDoZwrotu $gets$ 10
  return kwotaDoZwrotu
end
\end{algorithm}

\subsection{Faktury}
\subsubsection{Drukowanie faktur}
\begin{algorithm}[caption={2.2 Drukowanie faktur}, label={alg1}]
  input: daneDoFaktury
  output: faktura
  
begin
  faktura $gets$ FALSE
  if drukowanie(daneDoFaktury) == true 
    faktura $gets$ true
  else
    drukujFakture()
  return faktura
end
\end{algorithm}

\subsection{Instruktorzy}
\subsubsection{Rezerwacja instruktora}
\begin{algorithm}[caption={3.1 Rezerwacja instruktora}, label={alg1}]
  input: daneOsobowe, terminLekcji, idInstruktora
  output: lekcja
begin
  foreach idInstruktora
    if idInstruktora.terminLekcji == null 
      lekcja.id $gets$ daneOsobowe
      lekcja.idInstruktora $gets$ idInstruktora
      lekcja.termin $gets$ terminLekcji
      dodajLekcje(lekcja)
      obliczanieCenyZajec()
    else
      return message $gets$ "termin zajety"
  end
end
\end{algorithm}

\subsubsection{Obliczanie ceny zajęć}
\begin{algorithm}[caption={3.2 Obliczanie ceny zajęć}, label={alg1}]
  input: wybranaLekcja
  output: kwotaDoZaplaty

begin  
  koszt $gets$ pobierzKosztInstruktora(lekcja.idInstruktora)
  kwotaDoZaplaty $gets$ koszt
  wyborSposobuPlatnosci()
end
\end{algorithm}

\newpage
\subsection{Wypożyczalnia}
\subsubsection{Wynajęcie sprzętu. Zapisanie wypożyczenia}
\begin{algorithm}[caption={4.1.1 Wynajęcie sprzętu. Zapisanie wypożyczenia}, label={alg1}]
  input: wybranySprzet, idPracownika, idSprzetu
  output: wypozyczenie
  
begin
  wypozyczenie.idSprzetu $gets$ idSprzetu
  wypozyczenie.idPracownika $gets$ idPracownika
  stworzenieKoduWypozyczenia()
end
\end{algorithm}

\subsubsection{Wynajęcie sprzętu. Stworzenie kodu wypożyczenia}
\begin{algorithm}[caption={4.1.2 Wynajęcie sprzętu. Stworzenie kodu wypożyczenia}, label={alg1}]
  input: wypozyczenie, daneOsobowe
  output: idWypozyczenia
begin
  wypozyczenie.idWypozyczenia $gets$ md5(daneOsobowe + wypozyczenie.idSprzetu)
  dodajWypozyczenie(wypozyczenie)
  return wypozyczenie.idWypozyczenia
end
\end{algorithm}

\subsubsection{Obliczanie ceny wypożyczenia}
\begin{algorithm}[caption={4.2 Obliczanie ceny wypożyczenia}, label={alg1}]
  input: wybranySprzet
  output: kwotaDoZaplaty
begin
  koszt $gets$ pobierzKosztSprzetu(wybranySprzet)
  kwotaDoZaplaty $gets$ koszt
  wyborSposobuPlatnosci()
end
\end{algorithm}

\subsubsection{Przyjęcie zwracanego sprzętu}
\begin{algorithm}[caption={4.3 Przyjęcie zwracanego sprzętu}, label={alg1}]
  input: idWypozyczenia, idPracownika
  output: data
begin
  data $gets$ date.now
  dodajZwrotDoBazy(idWypozyczenia, data, idPracownika)
end
\end{algorithm}

\subsection{Serwis}
\subsubsection{Serwisowanie sprzętu. Przyjęcie zlecenia}
\begin{algorithm}[caption={5.1.1 Serwisowanie sprzętu. Przyjęcie zlecenia}, label={alg1}]

  input: wybranaUsluga, idPracownika, sprzet
  output: noweZlecenie
begin
  noweZlecenie.usluga $gets$ wybranaUsluga
  noweZlecenie.sprzet $gets$ sprzet
  noweZlecenie.idPracownika $gets$ idPracownika
  zapiszZlecenieDoBazy(noweZlecenie)
end
\end{algorithm}

\subsubsection{Serwisowanie sprzętu. Stworzenie kodu zlecenia}
\begin{algorithm}[caption={5.1.2 Serwisowanie sprzętu. Stworzenie kodu zlecenia}, label={alg1}]
  input: noweZlecenie, daneOsobowe
  output: idZlecenia
begin
  noweZlecenie.idZlecenia $gets$ md5(daneOsobowe + czas.now)
  dodajDoZleceniaSerwisu(noweZlecenie)
  return noweZlecenie.idZlecenia
end
\end{algorithm}

\subsubsection{Obliczenie ceny serwisowania}
\begin{algorithm}[caption={5.2 Obliczenie ceny serwisowania}, label={alg1}]
  input: wybraneUslugi
  output: kwotaDoZaplaty
  
begin
  suma $gets$ 0
  foreach usluga: wybraneUslugi
    koszt $gets$ pobierzKosztUslugi(wybranaUsluga)
    suma += koszt
  kwotaDoZaplaty $gets$ suma
  wyborSposobuPlatnosci()
end
\end{algorithm}

\subsubsection{Wydanie sprzętu po serwisowaniu}
\begin{algorithm}[caption={5.3 Wydanie sprzętu po serwisowaniu}, label={alg1}]
  input: idZlecenia, idPracownika
  output: sprzet
  
begin
  znajdzSprzet(idZlecenia)
  data $gets$ date.now
  dodajWydanieSprzetu(idZlecenia, data, idPracownika)
  wydajSprzet(sprzet)
end
\end{algorithm}

\subsection{Bramki}
\subsubsection{Identyfikacja karnetu}
\begin{algorithm}[caption={6.1 Identyfikacja karnetu}, label={alg1}]
  input: Karnet, aktywneKarnety
  output: dostep
  
begin
  foreach aktywneKarnety as k
    if karnet.id $gets$ k.id 
      if karnet.id == "*CZAS" 
        if karnet.odbicie+interwal<date.now 
          dostep $gets$ true
          zapisanieDatyIGodzinyOdbicia()
        else
          dostep $gets$ FALSE
      else if karnet.id == "*PKT" 
        if karnet.iloscPkt - kosztOdbicia > 0 
          dostep $gets$ true
          odjeciePunktow()
        else
          dostep $gets$ FALSE
      end if
    else
      dostep $gets$ FALSE
    end if
  end
end

\end{algorithm}

\subsubsection{Zapisanie daty i godziny odbicia}
\begin{algorithm}[caption={6.2 Zapisanie daty i godziny odbicia}, label={alg1}]
  input: karnet, dostep
  output: odbicie
  
begin
  karnet.odbicie $gets$ date.now
  zapiszOdbicieWBazie(karnet)
  return karnet.odbicie
end
\end{algorithm}

\subsubsection{Odjęcie punktów}
\begin{algorithm}[caption={6.3 Odjęcie punktów}, label={alg1}]
  input: karnet, bramka
  output: iloscPkt
begin
  karnet.iloscPkt $gets$ karnet.iloscPkt - bramka.kosztOdbicia
  return karnet
end
\end{algorithm}

\subsection{Płatność}
\subsubsection{Wybór sposobu zapłaty}
\begin{algorithm}[caption={7.1 Wybór sposobu zapłaty}, label={alg1}]
  input: sposobZaplaty, kwotaDoZaplaty
  output: kwotaDoZaplaty
begin
  if sposobZaplaty == "gotowka" 
    zaplataGotowka() 
  else if sposobZaplaty == "karta" 
    zaplataKarta()
  else if sposobZaplaty == "online" 
    zaplataOnline()
  end if
  return kwotaDoZaplaty
end
\end{algorithm}

\subsubsection{Zapłata gotówką}
\begin{algorithm}[caption={7.2.1 Zapłata gotówką}, label={alg1}]
  input: kwotaDoZaplaty
  output: zaplacono, idZaplaty
begin
  zaplacono $gets$ kwotaDoZaplaty
  idZaplaty $gets$ wygeneruj
  odnotowanieZaplaty()
end
\end{algorithm}

\subsubsection{Zapłata kartą}
\begin{algorithm}[caption={7.3.1 Zapłata kartą}, label={alg1}]
input: kwotaDoZaplaty
  output: zaplacono, idZaplaty
begin
  while pinNieprawidlowy and iloscProb > 0
    podajPin
    iloscProb--
  end
  if pinPrawidlowy 
    dokonajZaplaty(kwotaDoZaplaty)
    if dokonajZaplaty(kwotaDoZaplaty) == true 
      zaplacono =kwotaDoZaplaty
      idZaplaty $gets$ wygeneruj
      odnotowanieZaplaty()
    else
      return "odmowa dostepu"
    end if
  end if
end
\end{algorithm}

\subsubsection{Zapłata online}
\begin{algorithm}[caption={7.4.1 Zapłata online}, label={alg1}]
  input: kwotaDoZaplaty
  output: zaplacono, idZaplaty
begin
  polaczZApiDoPlatnosci
  dokonajZaplaty(kwotaDoZaplaty)
  if dokonajZaplaty(kwotaDoZaplaty) == true 
    zaplacono =kwotaDoZaplaty
    idZaplaty $gets$ wygeneruj
    odnotowanieZaplaty()
  else
    return message $gets$ "odmowa dostepu"
  end if
end
\end{algorithm}

\subsubsection{Odnotowanie zapłaty}
\begin{algorithm}[caption={7.5 Odnotowanie zapłaty}, label={alg1}]
  input: idZaplaty, zaplacono
  output: daneParagonu
begin
  ksiegujZaplate(idZaplaty, zaplacono)
  daneParagonu.id $gets$ kolejnePorzadkoweId
  daneParagonu.idZaplaty $gets$ idZaplaty
  daneParagonu.kwota $gets$ zaplacono
  drukowanieParagonu()
end
\end{algorithm}

\subsubsection{Drukowanie paragonu}
\begin{algorithm}[caption={7.6 Drukowanie paragonu}, label={alg1}]
  input: daneParagonu
  output: paragon
begin
  paragon $gets$ false
  druk $gets$ drukujParagon(daneParagonu)
  if druk == true 
    paragon $gets$ true
    return paragon
  else
    drukowanieParagonu
  end if
end
\end{algorithm}

