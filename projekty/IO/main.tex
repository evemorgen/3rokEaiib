\documentclass{sprawozdanie-agh}
\title{IO - dolina narciarska}
\usepackage{lscape}
\usepackage[final]{pdfpages}
\usepackage[utf8]{inputenc}
\usepackage{listings}
\usepackage{geometry}
\usepackage{lscape}
\geometry{
 a4paper,
 total={170mm,257mm},
 left=20mm,
 right=20mm,
 top=20mm,
}

\makeatletter

\begin{document}

\przedmiot{Inżynieria Oprogramowania}
\tytul{Projekt systemu \\ doliny narciarskiej \textit{Łojezusicku}}
\podtytul{...}
\kierunek{Informatyka}
\autor{Patryk Gałczyński, Szymon Duda}
\data{Kraków, 4 grudnia 2016}

\stronatytulowa{}

\section{Krótki opis systemu}
\large
Nasz projekt zakłada zamodelowanie działania kompleksu narciarskiego. Nasza stacja narciarska będzie oferować możliwość zakupu karnetów w kasie badź online. Klient będzie miał możliwość dokonania płatności gotówką, kartą lub przelewem. Do wyboru będzie miał różne metody potwierdzenia płatności, faktura, paragon. W naszej ofercie znajdą się karnety czasowe oraz "na punkty". Dla młodych narciarzy oraz grup zorganizowanych przygotowaliśmy specjalną ofertę zniżkową. Karnet czasowy umożliwia korzystanie ze wszystkich wyciągów w obrębie naszego kompleku w danym okresie, karnet punktowy pozwala na korzystanie z wyciągów do wyczerpania posiadanej ilości punktów na karnecie, przy czym, skorzystanie z określonego wyciągu redukuje ilość punktów na karcie o daną ilość, dla każdego wyciągu zdefiniowaną indywidualnie. Klient będzie musiał wypożyczyć kaucjonowaną kartę magnetyczną lub przypisać karnet do legitymacji z chipem. W ramach funkcjonowania całego kompleksu narciarskiego, klient będzie miał możliwość wynajęcia instruktora, oraz całego potrzebnego sprzętu w lokalnej wypożyczalni. W obrębie wypożyczalni będzie działać rówież serwis, gdzie klient będzie mógł zlecić renowację własnego sprzętu.

\end{document}
