\documentclass{sprawozdanie-agh}
\title{IO - dolina narciarska}
\usepackage{lscape}
\usepackage[final]{pdfpages}
\usepackage[utf8]{inputenc}
\usepackage{listings}
\usepackage{geometry}
\usepackage{lscape}
\usepackage{array}
\usepackage{float}
\usepackage{caption}
\captionsetup[table]{name=Tabela}
\geometry{
 a4paper,
 total={170mm,257mm},
 left=20mm,
 right=20mm,
 top=20mm,
}

\makeatletter

\begin{document}

\przedmiot{Inżynieria Oprogramowania}
\tytul{Projekt systemu \\ doliny narciarskiej \textit{Łojezusicku}}
\podtytul{...}
\kierunek{Informatyka}
\autor{Patryk Gałczyński, Szymon Duda}
\data{Kraków, 4 grudnia 2016}

\stronatytulowa{}

\tableofcontents

\newpage

\section{Krótki opis systemu}
\large
Nasz projekt zakłada zamodelowanie działania kompleksu narciarskiego. Nasza stacja narciarska będzie oferować możliwość zakupu karnetów w kasie badź online. Klient będzie miał możliwość dokonania płatności gotówką, kartą lub przelewem. Do wyboru będzie miał różne metody potwierdzenia płatności, faktura, paragon. W naszej ofercie znajdą się karnety czasowe oraz "na punkty". Dla młodych narciarzy oraz grup zorganizowanych przygotowaliśmy specjalną ofertę zniżkową. Karnet czasowy umożliwia korzystanie ze wszystkich wyciągów w obrębie naszego kompleku w danym okresie, karnet punktowy pozwala na korzystanie z wyciągów do wyczerpania posiadanej ilości punktów na karnecie, przy czym, skorzystanie z określonego wyciągu redukuje ilość punktów na karcie o daną ilość, dla każdego wyciągu zdefiniowaną indywidualnie. Klient będzie musiał wypożyczyć kaucjonowaną kartę magnetyczną lub przypisać karnet do legitymacji z chipem. W ramach funkcjonowania całego kompleksu narciarskiego, klient będzie miał możliwość wynajęcia instruktora, oraz całego potrzebnego sprzętu w lokalnej wypożyczalni. W obrębie wypożyczalni będzie działać rówież serwis, gdzie klient będzie mógł zlecić renowację własnego sprzętu.

\section{Lista bodźców zewnętrznych}
\begin{enumerate}
	\item żądanie kupna karnetu
	\item żądanie zwrotu nie używanego karnetu
	\item żądanie sprawdzenia stanu punktowego karnetu
	\item żądanie realizacji płatnośći
	\item żądanie wyboru metody potwierdzenia zakupu
	\item żądanie przypisania karnetu do karty (kaucjonowana lub legitymacja)
	\item żądanie zwrotu karty kaucjonowanej
	\item żądanie przeniesienia nieużytego karnetu na inny dzień
	\item żądanie rezerwacji instruktora
	\item żądanie wypożyczenia sprzetu
	\item żądanie oddania wypożyczonego sprzętu
	\item żądanie zlecenia serwis sprzętu
	\item żądanie zwrotu serwisowanego sprzętu
\end{enumerate}


\newpage
\section{Diagram przepływu danych}
\begin{center}
	\includegraphics[scale=0.25]{diagram_kontekstowy}
\end{center}

\begin{landscape}
	\newpage
	\section{Diagram przepływu danych - poziom 0}
	\begin{figure}
		\includepdf[trim=-3cm 0 0 0]{dfd/p0/p0-all-rotated-2}
	\end{figure}

	\newpage
	\section{Diagram przepływu danych - poziom 1}
	\subsection{Kasa biletowa}
	\begin{figure}
		\includepdf[trim=-3cm 0 0 0]{dfd/p1/p1-kasabiletowa-rotated}
	\end{figure}

	\newpage
	\subsection{Obsługa klienta}
	\begin{figure}
		\includepdf[trim=-3cm 0 0 0]{dfd/p1/p1-dzialobslugi-rotated}
	\end{figure}

	\newpage
	\subsection{Szkółka narciarska}
	\begin{figure}
		\includepdf[trim=-3cm 0 0 0]{dfd/p1/p1-szkola-rotated}
	\end{figure}

	\newpage
	\subsection{Wypożyczalnia sprzętu}
	\begin{figure}
		\includepdf[trim=-3cm 0 0 0]{dfd/p1/p1-wypozyczalnia-rotated}
	\end{figure}

	\newpage
	\subsection{Serwis sprzętu}
	\begin{figure}
		\includepdf[trim=-3cm 0 0 0]{dfd/p1/p1-serwis-rotated}
	\end{figure}

	\newpage
	\subsection{Bramki}
	\begin{figure}
		\includepdf[trim=-3cm 0 0 0]{dfd/p1/p1-bramki-rotated}
	\end{figure}

	\newpage
	\section{Diagram przepływu danych - poziom 2}
	\subsection{Kasa biletowa - kupno karnetu}
	\begin{figure}
		\includepdf[trim=-3cm 0 0 0]{dfd/p2/p2-kupnokarnetu-rotated}
	\end{figure}

	\newpage
	\subsection{Obsługa klienta - zwroty}
	\begin{figure}
		\includepdf[trim=-3cm 0 0 0]{dfd/p2/p2-zwroty-rotated}
	\end{figure}

	\newpage
	\subsection{Wypożyczalnia sprzętu - wypożyczenie}
	\begin{figure}
		\includepdf[trim=-3cm 0 0 0]{dfd/p2/p2-wypozyczenie-rotated}
	\end{figure}

	\newpage
	\subsection{Serwis sprzętu - serwisowanie}
	\begin{figure}
		\includepdf[trim=-3cm 0 0 0]{dfd/p2/p2-serwisowanie-rotated}
	\end{figure}

	\newpage
	\subsection{Płatność - gotówką}
	\begin{figure}
		\includepdf[trim=-3cm 0 0 0]{dfd/p2/p2-cash-rotated}
	\end{figure}

	\newpage
	\subsection{Płatność - kartą płatniczą}
	\begin{figure}
		\includepdf[trim=-3cm 0 0 0]{dfd/p2/p2-debit-rotated}
	\end{figure}

	\newpage
	\subsection{Płatność - online}
	\begin{figure}
		\includepdf[trim=-3cm 0 0 0]{dfd/p2/p2-online-rotated}
	\end{figure}

\end{landscape}
\newpage
\section{ERD}
\subsection{Diagram ERD}
\begin{figure}
	\includepdf[]{erd/erd}
\end{figure}

\newpage

\subsection{Klienci - opis}
\subsubsection{Klient}
\begin{table}[H]
	\centering
	\begin{tabular}{|c|c|c|c|}
		\hline
		Klucz & Nazwa atrybutu    & Typ danych & Obowiązkowy? \\ \hline
		PK    & id\_klienta       & integer    & tak           \\ \hline
		FK    & id\_karnetu       & integer    & nie           \\ \hline
		      & data\_rejestracji & datetime   & nie           \\ \hline
		      & nr\_dowodu        & text       & tak           \\ \hline
	\end{tabular}
	\caption{Klienci}
\end{table}

\subsubsection{Klient normalny}
\begin{table}[H]
	\centering
	\begin{tabular}{|c|c|c|c|}
		\hline
		Klucz & Nazwa atrybutu & Typ danych & Obowiązkowy? \\ \hline
		PK    & id\_klienta    & integer    & tak           \\ \hline
	\end{tabular}
	\caption{Klienci normalny}
\end{table}

\subsubsection{Klient zniżkowy}
\begin{table}[H]
	\centering
	\begin{tabular}{|c|c|c|c|}
		\hline
		Klucz & Nazwa atrybutu & Typ danych & Obowiązkowy? \\ \hline
		PK    & id\_klienta    & integer    & tak           \\ \hline
		FK    & id\_zniżki    & integer    & tak           \\ \hline
	\end{tabular}
	\caption{Klienci zniżkowy}
\end{table}

\subsection{Zniżki - opis}
\begin{table}[H]
	\centering
	\begin{tabular}{|c|c|c|c|}
		\hline
		Klucz & Nazwa atrybutu      & Typ danych & Obowiązkowy? \\ \hline
		PK    & id\_zniżki         & integer    & tak           \\ \hline
		      & wysokość\_zniżki & money      & tak           \\ \hline
		      & opis\_zniżki       & text       & tak           \\ \hline
		      & rabat\_grupowy      & boolean    & nie           \\ \hline
		      & rabat\_wiek         & boolean    & nie           \\ \hline
	\end{tabular}
	\caption{Zniżki}
\end{table}

\subsection{Pracownicy - opis}
\subsubsection{Pracownik}
\begin{table}[H]
	\centering
	\begin{tabular}{|c|c|c|c|}
		\hline
		Klucz & Nazwa atrybutu & Typ danych & Obowiązkowy? \\ \hline
		PK    & id\_pracownika & integer    & tak           \\ \hline
		FK    & id\_umowy      & integer    & tak           \\ \hline
		      & pesel          & text       & tak           \\ \hline
		      & nr\_dowodu     & text       & tak           \\ \hline
		      & imie           & text       & tak           \\ \hline
		      & nazwisko       & text       & tak           \\ \hline
	\end{tabular}
	\caption{Pracownicy}
\end{table}

\subsubsection{Pracownik serwisu}
\begin{table}[H]
	\centering
	\begin{tabular}{|c|c|c|c|}
		\hline
		Klucz & Nazwa atrybutu & Typ danych & Obowiązkowy? \\ \hline
		PK    & id\_pracownika & integer    & tak           \\ \hline
	\end{tabular}
	\caption{Pracownik serwisu}
\end{table}

\subsubsection{Pracownik wypożyczalni}
\begin{table}[H]
	\centering
	\begin{tabular}{|c|c|c|c|}
		\hline
		Klucz & Nazwa atrybutu & Typ danych & Obowiązkowy? \\ \hline
		PK    & id\_pracownika & integer    & tak           \\ \hline
	\end{tabular}
	\caption{Pracownik wypożyczalni}
\end{table}

\subsubsection{Pracownik kasy}
\begin{table}[H]
	\centering
	\begin{tabular}{|c|c|c|c|}
		\hline
		Klucz & Nazwa atrybutu & Typ danych & Obowiązkowy? \\ \hline
		PK    & id\_pracownika & integer    & tak           \\ \hline
	\end{tabular}
	\caption{Pracownik kasy}
\end{table}

\subsubsection{Pracownik działu obsługi}
\begin{table}[H]
	\centering
	\begin{tabular}{|c|c|c|c|}
		\hline
		Klucz & Nazwa atrybutu & Typ danych & Obowiązkowy? \\ \hline
		PK    & id\_pracownika & integer    & tak           \\ \hline
	\end{tabular}
	\caption{Pracownik działu obsługi}
\end{table}

\subsection{Instruktorzy - opis}
\subsubsection{Instruktor}
\begin{table}[H]
	\centering
	\begin{tabular}{|c|c|c|c|}
		\hline
		Klucz & Nazwa atrybutu & Typ danych & Obowiązkowy? \\ \hline
		PK    & id\_pracownika & integer    & tak           \\ \hline
		      & cena           & money      & tak           \\ \hline
	\end{tabular}
	\caption{Instruktor}
\end{table}

\subsubsection{Zajęcia instruktorów}
\begin{table}[H]
	\centering
	\begin{tabular}{|c|c|c|c|}
		\hline
		Klucz & Nazwa atrybutu    & Typ danych & Obowiązkowy? \\ \hline
		PK    & id\_zajęcia      & integer    & tak           \\ \hline
		FK    & id\_instruktora   & integer    & tak           \\ \hline
		      & ustalona\_godzina & datetime   & tak           \\ \hline
	\end{tabular}
	\caption{Zajęcia instruktorów}
\end{table}

\subsection{Umowy - opis}
\begin{table}[H]
	\centering
	\begin{tabular}{|c|c|c|c|}
		\hline
		Klucz & Nazwa atrybutu & Typ danych & Obowiązkowy? \\ \hline
		PK    & id\_umowy      & integer    & tak           \\ \hline
		      & rodzaj\_umowy  & text       & tak           \\ \hline
		      & wynagrodzenie  & money      & nie           \\ \hline
	\end{tabular}
	\caption{Umowy}
\end{table}

\subsection{Wypożyczenia - opis}
\subsubsection{Wypożyczenia}
\begin{table}[H]
	\centering
	\begin{tabular}{|c|c|c|c|}
		\hline
		Klucz & Nazwa atrybutu      & Typ danych & Obowiązkowy? \\ \hline
		PK    & id\_wypożyczenia   & integer    & tak           \\ \hline
		FK    & id\_sprzętu        & integer    & tak           \\ \hline
		FK    & id\_pracownika      & integer    & nie           \\ \hline
		FK    & id\_klienta         & integer    & nie           \\ \hline
		      & data\_wypożyczenia & datetime   & tak           \\ \hline
		      & data\_zwrotu        & datetime   & nie           \\ \hline
	\end{tabular}
	\caption{Wypożyczenia}
\end{table}

\subsubsection{Magazyn sprzętu}
\begin{table}[H]
	\centering
	\begin{tabular}{|c|c|c|c|}
		\hline
		Klucz & Nazwa atrybutu                        & Typ danych & Obowiązkowy? \\ \hline
		PK    & id\_sprzętu                          & integer    & tak           \\ \hline
		      & rodzaj\_sprzętu                      & text       & tak           \\ \hline
		      & cena\_wypożyczenia                   & money      & tak           \\ \hline
		      & maksymalna\_długość\_wypożyczenia & integer    & nie           \\ \hline
	\end{tabular}
	\caption{Magazyn sprzętu}
\end{table}

\subsection{Reklamacje i zwroty - opis}
\begin{table}[H]
	\centering
	\begin{tabular}{|c|c|c|c|}
		\hline
		Klucz & Nazwa atrybutu           & Typ danych & Obowiązkowy? \\ \hline
		PK    & id\_reklamacji           & integer    & tak           \\ \hline
		FK    & id\_pracownika\_obsługi & integer    & tak           \\ \hline
		FK    & id\_karnetu              & ineteger   & tak           \\ \hline
		      & opis\_reklamacji         & text       & nie           \\ \hline
	\end{tabular}
	\caption{Reklamacje i zwroty}
\end{table}

\subsection{Karnety - opis}
\subsubsection{Kupna karnetów}
\begin{table}[H]
	\centering
	\begin{tabular}{|c|c|c|c|}
		\hline
		Klucz & Nazwa atrybutu       & Typ danych & Obowiązkowy? \\ \hline
		PK    & id\_zakupu           & integer    & tak           \\ \hline
		FK    & id\_pracownika\_kasy & integer    & tak           \\ \hline
		FK    & id\_karnetu          & ineteger   & tak           \\ \hline
		      & cena\_końcowa       & money      & tak           \\ \hline
	\end{tabular}
	\caption{Kupna karnetów}
\end{table}

\subsubsection{Karnety}
\begin{table}[H]
	\centering
	\begin{tabular}{|c|c|c|c|}
		\hline
		Klucz & Nazwa atrybutu   & Typ danych & Obowiązkowy? \\ \hline
		PK    & id\_karnetu      & integer    & tak           \\ \hline
		      & cena\_podstawowa & money      & tak           \\ \hline
		      & opis\_karnetu    & text       & nie           \\ \hline
	\end{tabular}
	\caption{Karnety}
\end{table}

\subsubsection{Aktywne karnety}
\begin{table}[H]
	\centering
	\begin{tabular}{|c|c|c|c|}
		\hline
		Klucz & Nazwa atrybutu            & Typ danych & Obowiązkowy? \\ \hline
		PK    & id\_karnetu               & integer    & tak           \\ \hline
		FK    & id\_podstawowego\_karnetu & integer    & tak           \\ \hline
		      & data\_wydania             & datetime   & tak           \\ \hline
	\end{tabular}
	\caption{Aktywne karnety}
\end{table}

\subsubsection{Karnet punktowy}
\begin{table}[H]
	\centering
	\begin{tabular}{|c|c|c|c|}
		\hline
		Klucz & Nazwa atrybutu                & Typ danych & Obowiązkowy? \\ \hline
		PK    & id\_karnetu                   & integer    & tak           \\ \hline
		      & podstawowa\_ilość\_punktów & integer    & tak           \\ \hline
		      & pozostała\_ilosć\_punktów  & integer    & tak           \\ \hline
	\end{tabular}
	\caption{Karnet punktowy}
\end{table}

\subsubsection{Karnet czasowy}
\begin{table}[H]
	\centering
	\begin{tabular}{|c|c|c|c|}
		\hline
		Klucz & Nazwa atrybutu  & Typ danych & Obowiązkowy? \\ \hline
		PK    & id\_karnetu     & integer    & tak           \\ \hline
		      & czas\_aktywacji & datetime   & tak           \\ \hline
		      & ilość\_godzin & integer    & tak           \\ \hline
	\end{tabular}
	\caption{Karnet czasowy}
\end{table}

\subsection{Bramki - opis}
\subsubsection{Bramki}
\begin{table}[H]
	\centering
	\begin{tabular}{|c|c|c|c|}
		\hline
		Klucz & Nazwa atrybutu       & Typ danych & Obowiązkowy? \\ \hline
		PK    & id\_bramki           & integer    & tak           \\ \hline
		      & koszt\_punktowy      & integer    & tak           \\ \hline
		      & długość\_wyciągu & integer    & nie           \\ \hline
	\end{tabular}
	\caption{Bramki}
\end{table}

\subsubsection{Odbicia na bramkach}
\begin{table}[H]
	\centering
	\begin{tabular}{|c|c|c|c|}
		\hline
		Klucz & Nazwa atrybutu     & Typ danych & Obowiązkowy? \\ \hline
		PK    & id\_bramki         & integer    & tak           \\ \hline
		FK    & id\_karnetu        & integer    & tak           \\ \hline
		      & czas\_odbicia      & datetime   & tak           \\ \hline
		      & przyznany\_dostęp & boolean    & tak           \\ \hline
	\end{tabular}
	\caption{Odbicia na bramkach}
\end{table}

\subsection{Usługi - opis}
\subsubsection{Usługi}
\begin{table}[H]
	\centering
	\begin{tabular}{|c|c|c|c|}
		\hline
		Klucz & Nazwa atrybutu & Typ danych & Obowiązkowy? \\ \hline
		PK    & id\_usługi    & integer    & tak           \\ \hline
		      & opis\_usługi  & text       & tak           \\ \hline
		      & cena\_usługi  & money      & tak           \\ \hline
	\end{tabular}
	\caption{Usługi}
\end{table}

\subsubsection{Wykonane usługi}
\begin{table}[H]
	\centering
	\begin{tabular}{|c|c|c|c|}
		\hline
		Klucz & Nazwa atrybutu & Typ danych & Obowiązkowy? \\ \hline
		PK    & id\_wykonania  & integer    & tak           \\ \hline
		FK    & id\_usługi    & integer    & tak           \\ \hline
		FK    & id\_pracownika & integer    & tak           \\ \hline
	\end{tabular}
	\caption{Wykonane usługi}
\end{table}

\end{document}
